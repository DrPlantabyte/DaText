/*
This is a multi-line
comment. extra /* should not cause 
problems */

// single-line comment

# This is an annotation, a special 
# kind of comment that is parsed 
# by DaText and stored with the 
// interrupting an annotation with a comment shouldn't short-change the annotation
# key that is defined right after 

// extra new-lines should be okay

# the annotation. Comments may not 
# be preserved when a DaText file is
# read, changed, and rewritten, but 
# annotations will be preserved.
# Any character is alloed in an 
# annotation, even /* and //, which 
# ordinarily signify a comment
annotationKey // a key without a value is allowed, but means very little (can be used as a place-holder or flag)

# This is a standard key=value pair for text
text1=This is some text.
# The content of this string is escaped, it should appear as 
# Doug said "Lisa's teacher told me not to use / or \ in filenames."
text2 = Doug said \"Lisa\'s teacher told me not to use \/ or \\ in filenames.\"
# This should be identical to text2
text3 = "Doug said \"Lisa's teacher told me not to use / or \\ in filenames.\""
# This also should be identical to text2
text3 = 'Doug said "Lisa\'s teacher told me not to use / or \\ in filenames."'
/* This next key has not annotation (that's okay). Note that whitespace outside a
 * key or value is ignored, but whitespace inside is preserved (see String.trim() 
 * in the java API */
   key with spaces in name =     herp a derp

// You can (try to) parse any key as any value type. The following are demonstrations 
// of how that should work

# a base-10 integer
type is int10 = 12500
# a base-10 integer
type is int10 2 = 12 500
# a base-16 integer
type is hex = 0xEF1490
# a decimal, note that either . or , can mark the decimal place
type is float 1 = 12.345
# a decimal, note that either . or , can mark the decimal place
type is float 2 = 12,345
# you should use a whitespace as the thousands mark
type is float 3 = 12 345.67
# a decimal in sci notation
type is scifloat = 1.2345E+1 // locale sensitive
# a byte array (internal whitespace is ignored, but helps with human readability)
type is byte array = 01 23 45 67 89 AB CD EF 30 // note: this is hex format, NOT BASE64!
# not a number (text)
type is text = This is text that contains more then one line. \
See?
# not a number (text)
type is text 2 = "It is probably better to use quotes for multiline text.
See?
So much better!"

# object type is a DaText tree
object1={
	# a object inside an object
	object2={
		text1=heyyyy
		text2=derp
		number=5
	}
	# lists are semi-colon separated text values. intentional commas can be escaped or quoted
	list of names=[Bob; Bobby; Bobilina; 
		Bodacious; "Bob, slayer of bunnies; just ignore him"; 
		Herp\; the derper;;;;;;] // trailing semicolon should not cause errors, empty values not added to list
}
 
/*
That's all the DaText types (int, hex int, decimal, byte array, text, objects, and lists.
No more will be added to the spec for two reasons:
	1. Adding any more will make DaText more complicated and difficult to read 
and implement (defeating the purpose of DaText)
	2. All further extensions can be made by either storing other types as text 
or as binary data
Note that there is no schema or other validation for DaText (if you need that, 
switch to XML as your requirements fall outside the intention of DaText). The 
type o data is determined only by which method is used to parse the DaText 
ket=value pair, so putting text in a key that "should be a number" is a valid 
DaText file, but will throw a NumberFormatException when the program tries to 
use that key as a number (the exception's message should point to where in the 
file the problem key is defined).
*/
